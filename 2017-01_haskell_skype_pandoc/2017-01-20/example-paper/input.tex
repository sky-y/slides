%%%%%%%%%%%%%%%%%%%%%%%%%%%%%%%%%%%%%%%%%%%%%%%%%%%%%%%%%%%%%%%%%%%%
%%  研究リテラシー「TeX入門」
%%  酒井 高司
%%  2016/05/18
%%%%%%%%%%%%%%%%%%%%%%%%%%%%%%%%%%%%%%%%%%%%%%%%%%%%%%%%%%%%%%%%%%%%
\documentclass[11pt]{jarticle} %jarticleクラスを使用,文字サイズを11ポイントに設定
\setlength{\textheight}{23cm} %本文領域の高さ(縦幅)を23cmに設定
\setlength{\textwidth}{16cm} %本文領域の横幅を16cmに設定
\setlength{\oddsidemargin}{0cm} %左端の余白をデフォルトから0cmに設定
\setlength{\topmargin}{0cm} %上端の余白をデフォルトから0cmに設定
\renewcommand{\baselinestretch}{1.2} %行間をデフォルトの1.2倍に設定
\title{\TeX 入門} %タイトルを記入
\author{酒井 高司\\(首都大学東京理工学研究科数理情報科学専攻)} %著者名を記入
\date{\today} %日付を記入
\begin{document}
\maketitle %この位置にタイトルを作成
\begin{abstract} %ここに概要を記入
\LaTeX の使い方の解説です.
\end{abstract}

\tableofcontents %この位置に目次を作成

\section{\TeX について} %セクションを作成

\TeX (「テフ」と読む)はDonald Ervin Knuth氏が製作した組版システムです.
現在ではKnuth氏が作った\TeX をもとにたくさんのバージョンが存在します.
講義では現在一般的に用いられている日本語\TeX である
p\LaTeXe (ピー・ラテフ・トゥー・イー)の使い方を学びます.
詳しい使い方については\cite{Okumura,Yoshinaga,Shimizu}を参照して下さい.

\subsection{\TeX の特徴} %セクションの中にサブセクションを作成

TeXには次のような特徴があります.
\begin{itemize}
\item フリーウェアである.ソースコードも公開されている.
\item Windows, Macintosh, LinuxなどほとんどすべてのOS上で動作する.
\item 数式を表示する機能に優れている.複雑な数式もきれいに出力させることができる.
\item 式番号や文献の参照の管理を自動的に行ってくれる.
\item 文章のレイアウトを自動的に決めてくれる.
一方で自分で指定する自由度も高く,レイアウトを細かく指定することもできる.
\item 標準化が徹底されており,高い再現性をもつ.
\item テキスト形式で入力するため,データを長期的に利用できる.
\end{itemize}

\begin{thebibliography}{9} %この位置に文献表を作成
\bibitem{Okumura} 奥村晴彦「\LaTeXe 美文書作成入門」(技術評論社)
\bibitem{Yoshinaga} 吉永徹美「独習\LaTeXe 」(翔泳社)
\bibitem{Shimizu} 清水美樹「はじめての\LaTeX 」(工学社)
\end{thebibliography}
\end{document}
